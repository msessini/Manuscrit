\addchap{Introduction}

À l'heure de l'écriture de cette thèse, plus de 11 années se sont écoulées depuis l'annonce de la découverte du boson de Higgs par les collaborations ATLAS et CMS au CERN le 4 juillet 2012 \cite{ATLASdiscovery,CMSdiscovery}. Sa découverte frappe notamment par la force avec laquelle elle vient affirmer le modèle théorique ayant prédit son existence près de 50 ans plus tôt \cite{BE,H}, et marque l'observation de la dernière particule élémentaire encore non observée prédite par le modèle standard de la physique des particules \cite{SM1,SM2,SM3}. Une vaste partie du programme de physique des hautes énergies est depuis consacrée à la mesure de précision des propriétés du boson de Higgs. Dans cette thèse, ses propriétés de charge-parité (CP) à travers son couplage au lepton tau sont étudiées. À ce jour, l'étude des couplages du boson de Higgs aux bosons vecteurs se montre largement en faveur de l'hypothèse d'un boson de Higgs de nature purement scalaire tel que prédit par le modèle standard, et semble exclure son éventuelle nature pseudo-scalaire. En revanche, l'apparition d'une violation de CP est toujours possible au premier ordre dans les couplages de Yukawa, à travers un angle de mélange faisant intervenir à la fois une contribution scalaire et une contribution pseudo-scalaire au sein de ce dernier. Les premières mesures de cet angle de mélange dans plusieurs couplages \cite{ttH,Htautau} avec les données du Run 2 restent en accord avec l'hypothèse purement scalaire mais peuvent encore bénéficier, notamment dans le cas du couplage au lepton tau, de l'accroissement du volume de données produites dans les prochaines phases d'exploitation du \textit{Large Hadron Collider} (LHC) et d'une optimisation des méthodes de mesure. Toute découverte de violation de CP dans ce secteur constituerait alors une preuve tangible de l'existence de nouvelle physique, avec d'éventuelles implications en cosmologie dans le cadre de la baryogénèse électrofaible \cite{EWKbaryogenesis} en particulier. \\

Le premier chapitre de cette thèse sera dédié à une introduction historique des concepts de la physique moderne du début du $XX^e$ siècle et ayant donné lieu aux bases théoriques et expérimentales de la description du monde quantique. \\

Le second chapitre présentera ensuite le modèle standard de la physique des particules, en donnant des éléments théoriques de la structure des différentes interactions fondamentales qu'il décrit ainsi que du mécanisme de Brout-Englert-Higgs (BEH) ayant conduit à la brisure de symétrie électrofaible. \\

Dans le troisième chapitre, une présentation du CERN et de son histoire ainsi qu'une description du LHC et du détecteur CMS (\textit{Compact Muon Solenoid}) seront données. \\

Le quatrième chapitre contient une description des méthodes d'identification et de reconstruction des différentes particules et objets avec le détecteur CMS, et dans lequel un accent particulier est porté sur le lepton tau avec une étude sur les performances d'identification réalisée dans le cadre des EPR (\textit{Experimental Physics Responsabilities}) durant cette thèse. \\

Le cinquième chapitre est un état de l'art concernant la mesure expérimentale des propriétés du boson de Higgs et la recherche de violation de CP dans le secteur du Higgs, avec un regroupement des résultats de plusieurs analyses réalisées par la collaboration CMS. \\

Le sixième chapitre porte sur l'introduction des méthodes expérimentales de mesure de l'état CP du boson de Higgs employées au LHC. Parmi celles-ci se trouve la méthode du vecteur polarimétrique, dont les performances sont étudiées dans les canaux hadroniques $H\to\tau_h\tau_h$ et semi-leptoniques $H\to\mu\tau_h$. D'autres méthodes, propres à l'analyse de la structure CP du couplage de Yukawa du lepton tau sont également introduites avec notamment la méthode d'$embedding$ servant à la description du bruit de fond $Z\to\tau\tau$. Une étude des variables optimales de spin du lepton tau dans ces échantillons est aussi présentée. \\

Enfin, le septième chapitre présente le déroulement d'une mesure de l'angle de mélange CP dans le canal $a_1^{3pr}+\mu$ avec les données du Run 2. Dans cette analyse, la méthode du vecteur polarimétrique est comparée aux méthodes initiales avec une recherche d'optimisation de la sensibilité.