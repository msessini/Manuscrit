\addchap{Conclusion}

L'étude principale présentée dans cette thèse est portée sur la recherche de violation de CP dans le couplage de Yukawa du lepton tau avec la méthode du vecteur polarimétrique dans les données du Run 2 de l'expérience CMS. Ces travaux ont démarré en 2020, lors des dernières étapes de l'analyse de la structure CP de ce même couplage dont les résultats ont été publiés en 2021 \cite{Htautau}. Cette publication intègre notamment le déploiement pour la première fois de la méthode du vecteur polarimétrique dans le canal hadronique $a_1^{3pr}a_1^{3pr}$, suite aux travaux de thèse de Guillaume Bourgatte à l'Institut pluridisciplinaire Hubert Curien \cite{guigui}. Les bonnes performances de cette méthode face à la méthode originale utilisant seulement l'information visible de la désintégration des leptons tau pour la reconstruction de l'observable $\phi_{CP}$ ont motivé à étudier les possibilités de son déploiement dans d'autres canaux. Une majeure partie de cette thèse a ainsi été consacrée à cette tâche, en réalisant une étude des performances de la méthode du vecteur polarimétrique dans les canaux hadroniques $\tau_h\tau_h$ et semi-leptoniques $\mu\tau_h$. Le code d'analyse strasbourgeois ayant été initialement développé uniquement pour l'étude des états finals purement hadroniques, l'étude des états finals comportant un muon a requis la mise en place de la procédure de sélection de la paire $\mu\tau_h$, avec les coupures cinématiques et les chemins du système de déclenchement qui lui sont propres. Cette étude a permis de mettre en lumière une possibilité d'amélioration de la sensibilité de mesure de l'état CP du boson de Higgs par l'emploi de la méthode du vecteur polarimétrique sur tous les états finals hadroniques comportant au moins un hadron neutre et un ou plusieurs hadrons chargés. Les améliorations sont toutefois limitées par les performances actuelles de reconstruction de l'impulsion totale des leptons taus, et des grandeurs angulaires en particulier. En parallèle, un stage mené par deux étudiantes de Master 1 de l'Université de Strasbourg, encadrées par Anne-Catherine Le Bihan et moi-même, a permis de confirmer que la méthode du vecteur polarimétrique n'introduit pas de biais dans la reconstruction de $\phi_{CP}$ dans le bruit de fond $Z\to\tau\tau$. Cette étude a été menée à la fois sur des échantillons Monte Carlo et \textit{embedded}. \\

Les canaux impliquant au moins une résonance $a_1^{3pr}$ bénéficient toutefois d'une contrainte supplémentaire sur la direction du lepton tau grâce à la mesure du vertex secondaire. L'algorithme GEF \cite{GEF} permet, dans ce cas de figure, d'effectuer une reconstruction complète de l'évènement $H/Z\to\tau\tau$ par un ajustement cinématique. Le canal $a_1^{3pr}+\mu$ a ainsi été choisi pour réaliser une analyse complète sur les données du Run 2 avec une mesure de l'angle de mélange CP grâce à la méthode du vecteur polarimétrique. Cette mesure a d'abord nécessité une optimisation du traitement des erreurs systématiques au sein du framework d'analyse et l'insertion des corrections des échantillons Monte Carlo adaptées à ce canal, puis la production des données sur la Worldwide LHC Computing Grid (WLCG) grâce aux outils de production de CMS et leur analyse sur la grille de calcul locale Tier-2 de Strasbourg. Un BDT destiné à effectuer une classification des évènements et produire des catégories distinctes enrichies en bruit de fond ou en signal a été développé pour les besoins de cette mesure. Les résultats montrent une amélioration de l'ordre de $15\%$ de la sensibilité attendue sur le Run 2 avec une exclusion de l'hypothèse CP impaire contre l'hypothèse CP paire à $1,06\sigma$ par la méthode du vecteur polarimétrique. Lors de la procédure d'\textit{unblinding}, réalisée sur les données de 2018 pour lesquelles le BDT a été entraîné, le test de qualité d'ajustement avec un modèle saturé a montré une incompatibilité entre le modèle de signal obtenu par la méthode du vecteur polarimétrique et les données. Une bonne compatibilité est toutefois observée avec les modèles de bruit de fond, en particulier pour le bruit de fond $Z\to\tau\tau$ et tous les processus comprenant deux leptons taus correctement identifiés avec une valeur de probabilité $p>0,8$. Ces résultats ont conduit à ne dévoiler que les données de 2018 pour la méthode de la decay plane, dans lesquelles les fluctuations statistiques ne permettent pas d'obtenir une sensibilité à la mesure de l'état CP. Après investigation, une erreur de modélisation du vertex secondaire du lepton tau, utilisé par l'algorithme GEF, semble être une cause probable de la mauvaise modélisation de l'observable $\phi_{CP}$ avec la méthode du vecteur polarimétrique. \\

Plusieurs tâches ont également été réalisées au cours de cette thèse dans le cadre des EPR proposés par l'expérience CMS. Une d'elle a consisté a vérifier l'impact de l'identification des électrons et des photons sur l'identification et la reconstruction des leptons tau. L'algorithme du flux de particules intègre un réseau de neurones profond en charge de l'identification des électrons et des photons, et dont l'entraînement a été reproduit en vue du démarrage du Run 3. Après la fin du Run 1, il a été remarqué que cette procédure était à l'origine d'une perte d'efficacité de l'ordre de $5-10\%$ dans la reconstruction des modes de désintégration $\tau_h\to\pi^{\pm}\pi^0s$. Cette perte est associée à la reconstruction des photons convertis en paire électron/positron à partir d'une seule trace, entraînant un accroissement du taux de hadrons faussement identifiés en tant que photons. Cette étude, menée conjointement avec les groupes E/Gamma et Tau de CMS, a permis de s'assurer que le nouvel entraînement du réseau de neurone n'entraîne pas de perte d'efficacité supplémentaire sur l'identification et la reconstruction des taus avant son intégration officielle dans le software de CMS. Un second EPR a également été réalisé dans le but de vérifier la distribution des variables optimales de spin du lepton tau dans la désintégration $Z\to\tau\tau$ dans des échantillons \textit{embedded}. Ces variables jouent un rôle crucial dans diverses analyses qui requièrent une sensibilité à l'état de spin du lepton tau, notamment dans la mesure de sa polarisation \cite{Zpol} et font également intervenir le concept de vecteur polarimétrique. Le déploiement progressif des échantillons \textit{embedded} dans les analyses demande ainsi de s'assurer de la bonne modélisation de ces variables. Enfin, ces trois années de thèse furent également l'occasion de joindre mes efforts à diverses activités telle que la coordination de la production des échantillons Monte Carlo pour le Run 3 dans lesquels des leptons taus sont impliqués, ou la réalisation de shifts en tant que \textit{trigger shifter} afin de s'assurer du bon fonctionnement du système de déclenchement lors des prises de données.