\begin{titlepage}

      \placetextbox{0.15}{0.989}{\includegraphics[width=0.22\paperwidth]{Images/Signature_Universite_Strasbourg_Unistra2_Transparent.png}}
 
    
      \begin{center}
      
      
      \vspace*{-1.2cm} %CHANGED
      
      {\Large \hspace{0.6cm} \textbf{UNIVERSITÉ DE STRASBOURG}}\\
      \vspace*{1.5cm} %CHANGED
      
      {\large \textbf{ÉCOLE DOCTORALE DE PHYSIQUE ET CHIMIE PHYSIQUE}}\\
      \textbf{Institut Pluridisciplinaire Hubert Curien (IPHC), UMR 7178}\\
      
      \vspace*{1.5cm} %CHANGED
      
      {\LARGE \textbf{TH\`{E}SE}}
      
      présentée par:
      
      \vspace*{0.3cm}
      
      {\Large \textbf{Mario Sessini}}
      
      soutenue le : \textbf{15 décembre 2023}
      
      
      \vspace*{0.5cm}
      
      pour obtenir le grade de: \textbf{Docteur de l'Université de Strasbourg}
      
      
      Discipline/Spécialité: Physique des particules élémentaires
      
      
      \vspace*{1cm}
      
      \fbox{
            \parbox{\textwidth}{\centering \LARGE \textbf{Recherche de violation de CP dans le couplage de Yukawa du lepton $\tau$ avec la méthode du vecteur polarimétrique dans les données du Run 2 de l'expérience CMS auprès du LHC}} %}
            %\centering
      }
      
      \end{center}
      
       \small
      
       \vspace*{0.2cm}
       %\vspace*{0.8cm}
      
       {\large \textbf{THÈSE dirigée par:}} %\\
      
       \vspace*{0.5cm}
      
       \setlength{\tabcolsep}{0.5cm}
       \begin{tabular}{ll}
             \textbf{Dr. Anne-Catherine LE BIHAN}          & Institut pluridisciplinaire Hubert Curien\\
       \end{tabular}
      
       \hrulefill
      
       \vspace*{0.5cm}
      
      
      % \begin{doublespace}
      
      
             {\large \textbf{RAPPORTEURS:}} %\\
      
             \vspace*{0.5cm}
      
             \setlength{\tabcolsep}{.5cm}
             \begin{tabular}{ll}
                   \textbf{Dr. Marco DELMASTRO} & Laboratoire d'Annecy de physique des particules\\
                   \textbf{Dr. Dirk ZERWAS} & Laboratoire de physique des 2 infinis Irène Joliot-Curie \\
             \end{tabular}
      
             \vspace*{0.8cm}
      
             {\large \textbf{EXAMINATEUR ET EXAMINATRICE:}} %\\
      
             \vspace*{0.5cm}
      
             \setlength{\tabcolsep}{0.5cm}
             \begin{tabular}{ll}
                  \textbf{Dr. Florian BEAUDETTE}          &  Laboratoire Leprince-Ringuet\\
                  \textbf{Dr. Isabelle RIPP-BAUDOT}          & Institut pluridisciplinaire Hubert Curien\\

             \end{tabular}
      
      % \end{doublespace}
      
      
      \end{titlepage}